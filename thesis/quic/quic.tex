\section{Motivacija za nečim \textit{boljim} od \textit{\textlatin{TCP}}}

Porast broja servisa \textit{osjetljivih na latentnost mreže}, kao i želja za sveobuhvatno boljim korisničkim iskustvom, inspirisala je motivaciju za ponovnim razmatranjem transportnih protokola. Međutim, pokušaji direktne modifikacije protokola radi redukcije latentnosti na transportnom sloju često je nailazio na sljedeće prepreke
\begin{enumerate}
    \item \textbf{Implementaciona težina}. \textit{\textlatin{TCP}} protokol često je sastav samog operativnog sistema (dakle, egzistira u tzv. \textit{\textlatin{kernel}} prostoru), što će značiti da bilo kakva izmjena samog protokola iziskuje ažuriranje operativnog sistema. Ovo, naravno, nije optimalno rješenje jer je neskalabilno, a često i neizvodljivo.
    \item \textbf{\enquote{Ukorijenjenost} \textit{\textlatin{TCP}} protokola}. Pojam \textit{osifikacije} (ili, \textit{ukorijenjenosti}) protokola označava preveliko oslanjanje na konkretna svojstva protokola od strane \textit{\textlatin{middlebox}} uređaja\footnote{\textit{\textlatin{Middlebox}} uređaji su mrežni uređaji koji vrše inspekciju, transformaciju ili filtriranje određenih dijelova paketa zarad sigurnosti, transliranja adresa i sl.}, pa sama modifikacija datih je praktično nemoguća. Pomenuti problem često se pojavljuje na raznoraznim nivoima mrežne komunikacije, a konkretno za \textit{\textlatin{TCP}} i urkos pokušaju predstavljanja \textit{\textlatin{TCP Fast Open}} ekstenzije, mnoge konfiguracije se još oslanjaju na originalnu \textit{\textlatin{TCP}} implementaciju.
    \item \textbf{Latentnost koju unosi \textit{\textlatin{handshake}} proces}.
    \item \textbf{\textit{\textlatin{Head-of-line}} problem}.
\end{enumerate}
Generalno, uviđa se da se rješenje problema latentnosti \textit{\textlatin{TCP}} protokola ne može isključivo ostvariti na implementacionom dijelu. Stoga, pribjeglo se dizajniranju protokola sa mišlju zadržavanjem robusnosti pomenutog protokola, a u isto vrijeme i pružanju nezavisnosti i pomjeraja kontrole transporta u aplikativni sloj, te time nastaje \textit{\textbf{\textlatin{QUIC}}} protokol. 

\section{\textit{\textlatin{QUIC}} protokol}

\textit{\textbf{\textlatin{QUIC}}} \textbf{mrežni protokol} predstavlja protokol opšte namjene razvijen od strane kompanije \textit{Gugl} (engl., \textit{\textlatin{Google}}) i detaljno predstavljen u radu \cite{46403}. Dati protokol pojavljivanjem donosi poboljšanje performansi mrežnih servisa, prevashodno onih \textit{osjetljivih na latenciju}.    

U kontekstu (enkriptovane) \textit{\textlatin{HTTP}} komunikacije, \textit{\textlatin{QUIC}} protokol pokušava premostiti ograničenja \textit{\textlatin{TCP}} protokola koje se tiču latentnosti i oporavka od greške. Primarno, \textit{\textlatin{QUIC}} pomenuto uspjeva na dva načina
\begin{itemize}
    \item \textbf{Redukcija \textit{troška} u komunikaciji kroz inicijalnu fazu povezivanja}. Kako većina \textit{\textlatin{HTTP}} konekcija zahtjeva \textit{\textlatin{TLS}} danas, \textit{\textlatin{QUIC}} protokol tokom inicijalnog \textit{\textlatin{handshake}} procesa (inicijalog otvaranja konekcije), \textbf{prirodaje i potrebne ključeve za uspostavljanje sigurne konekcije}. Ovakav način dejstvovanja ogromno redukuje potrebu konstantnog \textit{pregovora} oko sigurnosnog protokola, jer se sigurnosni podaci \textit{čuvaju} za svaki narednu komunikaciju dvaju strana.
    \item \textbf{Korištenje \textit{\textlatin{UDP}} umjesto \textbf{\textit{\textlatin{TCP}}} protokola kao substrat}. Međutim, kao što je i prethodno pomenuto, \textit{\textlatin{UDP}} protokol ne sadrži oporavak od gubitka paketa. Kako bi \textit{\textlatin{QUIC}} prevazišao ovaj nedostatak, \textit{retransmisija izgubljenih paketa} je implementirana na \textit{\textlatin{QUIC}} sloju, umjesto na \textit{\textlatin{UDP}} sloju. Takođe, \textit{\textlatin{QUIC}} \enquote{otvara} \textit{međusobno odvojene kanale komunikacije}, što omogućava \textbf{neblokirajuću} operaciju slanja paketa i u slučaju kada se pojavi potreba za retransmisijom podataka, za razliku od \textit{\textlatin{TCP}} protokola.
\end{itemize}
\begin{figure}[!ht]
    \centering
    \includegraphics[scale=0.2]{images/tcp-vs-quic-handshake.png}
    \caption{Razlika u procesu uspostavljanja komunikacije između \textit{\textlatin{TCP}} i \textit{\textlatin{QUIC}} protokola}
\end{figure}
