\section{Osnovni mrežni protokoli na transportnom nivou}

\subsection{\textit{\textlatin{TCP}} mrežni protokol}
\textit{\textbf{\textlatin{TCP}}} \textbf{protokol} predstavlja osnovu internet komunikacije današnjeg doba. Protokol ima za cilj davanje interfejsa pri komunikaciji među dva čvora u mreži. Podaci se, kroz \textit{\textlatin{TCP}} protokol, šalju \textit{u oktetnim paketima}, i to na \textbf{pouzdan}, \textbf{uređen način}. Naime, svaki od paketa je markiran \textit{rednim brojem u sekvenci paketa}, kao i \textit{kontrolnu sumu}, a ako se desi \textit{nestanak jednog} ili \textit{greška u jednom od datih}, sam protokol konstantuje \textit{neregularno stanje}. U ovom slučaju, \textit{\textlatin{TCP}} koristi \textit{\textlatin{ARQ}} (akronim za \textit{\textlatin{automatic repeat request}}, engl.) mehanizam, gdje pokušava da \textit{ponovo pošalje podatke}.

U većini implementacija \textit{\textlatin{TCP}} protokola, kada se detektuje greška u komunikaciji, poziva se \textit{blokirajuća operacija} gdje se svi naredni transferi \textit{zaustavljaju} \textit{ili dok ne dođe do uspješnog ponovnog slanja paketa}, \textit{ili dok se konekcija ne prekine}. To je posljedica činjenice da se \textit{\textlatin{TCP}} protokolom uspostavlja isključivo \textit{jedan kanal} (engl. \textit{pipe}, \textit{stream}), te je jasno da \textit{performanse komunikacije bivaju degradirane} (kao što je slučaj kod \textit{\textlatin{HTTP}}/2 protokola\footnote{Iako je \textit{\textlatin{HTTP}}/2 protokol uveo više kanala, to je isključivo implementacija na aplikativnom nivou - u pozadini se oni \textit{multipleksiraju} na jednom \textit{\textlatin{TCP}} kanalu. Svakako, ovo predstavlja poboljšanje u odnosu na \textit{\textlatin{HTTP}}/1 protokol.}, koji se, kao što je poznato, oslanja na \textit{\textlatin{TCP}}). Ovaj problem u literaturi je poznat kao \textit{\textlatin{head-of-line blocking}}.

Pomenuti \textit{\textlatin{TCP}} protokol \textit{namjerno mimikuje} \textit{\textlatin{data}}-\textit{\textlatin{pipe}} i kao takav poznaje malo informacija o podacima koje prenosi. Usljed inherentnog trajanja prenosa u komunikaciji između udaljenih čvorova u mreži, svaki dodatan zahtjev nad podacima (kao što je, recimo, \textit{enkripcija} koristeći \textit{\textlatin{TLS}}) unosi dodatnu latentnost, kako je potrebno više \textit{razmjena} radi ostvarivanja konekcije.

\subsection{Uspostavljanje konekcije između čvorova kroz \textit{\textlatin{TCP}} protokol}

Operacije \textit{\textlatin{TCP}} protokola mogu biti podjeljenje u tri faze
\begin{itemize}
    \item \textbf{Uspostavljanje konekcije} između dva čvora, što uključuje višekoračni proces \enquote{rukovanja} (engl. \textit{\textlatin{handshake}}).
    \item \textbf{Period transfera podataka}.
    \item \textbf{Prekid konekcije}, nakon završetka prenosa podataka, nakon čega dolazi do oslobađanja zauzetih resursa na čvorovima.
\end{itemize}

\subsection{\textit{\textlatin{UDP}} mrežni protokol}

\section{\textit{\textlatin{QUIC}} mrežni protokol kao poboljšanje \textit{\textlatin{TCP}} protokola}

\textit{\textbf{\textlatin{QUIC}}} \textbf{mrežni protokol} predstavlja protokol opšte namjene koji \textit{dejstvuje na transportnom sloju}, razvijen od strane kompanije \textit{Gugl} (engl., \textit{\textlatin{Google}}). Dati protokol je pojavljivanjem donio poboljšanje performansi mrežnih aplikacija, koje su se prevashodno oslanjale na klasični \textit{\textlatin{TCP}} protokol.

U kontekstu (enkriptovane) \textit{\textlatin{HTTP}} komunikacije, \textit{\textlatin{QUIC}} protokol pokušava premostiti ograničenja \textit{\textlatin{TCP}} protokola koje se tiču latentnosti i oporavka od greške. Primarno, \textit{\textlatin{QUIC}} pomenuto uspjeva na dva načina
\begin{itemize}
    \item \textbf{Redukcija \textit{troška} u komunikaciji kroz inicijalnu fazu povezivanja}. Kako većina \textit{\textlatin{HTTP}} konekcija zahtjeva \textit{TLS} danas, \textit{\textlatin{QUIC}} protokol tokom inicijalnog \textit{handshake} procesa (inicijalog otvaranja konekcije), \textbf{prirodaje i potrebne ključeve za uspostavljanje sigurne konekcije}. Ovakav način dejstvovanja ogromno redukuje potrebu konstantnog \textit{pregovora} oko sigurnosnog protokola, jer se sigurnosni podaci \textit{čuvaju} za svaki narednu komunikaciju dvaju strana.
    \item \textbf{Korištenje \textit{\textlatin{UDP}} umjesto \textbf{\textlatin{TCP}} protokola kao osnovu}. Međutim, kao što je i prethodno pomenuto, \textit{\textlatin{UDP}} protokol ne sadrži oporavak od gubitka paketa. Kako bi \textit{\textlatin{QUIC}} prevazišao ovaj nedostatak, \textit{retransmisija izgubljenih paketa} je implementirana na \textit{\textlatin{QUIC}} sloju, umjesto na \textit{\textlatin{UDP}} sloju. Takođe, \textit{\textlatin{QUIC}} \enquote{otvara} \textit{međusobno odvojene kanale komunikacije}, što omogućava neblokirajuću operaciju slanja paketa i u slučaju kada se pojavi potreba za retransmisijom podataka, za razliku od \textit{\textlatin{TCP}} protokola.
\end{itemize}

