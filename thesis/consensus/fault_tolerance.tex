\section{Tolerancija na otkaze}
Karakteristično svojstvo distribuiranih sistema koji ih razlikuju od sistema na jednoj mašini jeste \textit{pojam o djelimičnom otkazu} - prihvatljivost činjenice da dio sistema ne funkcioniše onako \textit{kako bi trebalo} (\textit{kako je zamišljeno}) a da sistem daje predstavu kao da je \enquote{sve u redu}. Pod pojmom \textit{otkaza}, pokušavamo dati abstrakciju nad time \enquote{šta to loše može poći u sistemu} - komunikaciona mreža ne funkcioniše, čvorovi su krahirali ili se ponašaju maliciozno su neki od primjera. Generalno, pomenutu apstrakciju otkaza klasifikujemo kao
\begin{enumerate}
    \item \textit{tranzijentnu}, gdje se dati otkazi dešavaju isključivo jednom, te nestanu u potpunosti.
    \item \textit{ponovljenu}, koji se dešavaju repetitivno u određenim periodama vremena.
    \item \textit{permanentnu}, gdje otkazi postoje sve dok komponenta koja ih ne uzrokuje bude otklonjena.
\end{enumerate}

Izgradnja distribuiranog sistema usko je povezana sa kontrolom otkaza. Naime, postoji distinkcija između prevencije, tolerancije, otklanjanja i previda otkaza, a svaka od pomenutih akcija mora biti \enquote{dobro} implementirane i kontrolisane kako bi sistem nesmetano funkcionisao. Od našeg najvećeg interesa jeste osobina \textbf{tolerancije na otkaze}, koja, ako implementirana, \textbf{daje sistemu osobinu funkcionisanja i pružanja servisa i u slučaju kada postoje otkazi}.

Tolerancija na otkaze je tema koja se decenijama prožima kroz naučno-istraživački rad na polju računarstva. Definišu se četiri osobine pod kojima se smatra da je distribuiran sistem \textit{\textbf{tolerantan na otkaze}}, a to su
\begin{enumerate}
    \item \textbf{Dostupnost} (engl. \textit{\textlatin{availability}})
    \item \textbf{Pouzdanost} (engl. \textit{\textlatin{reliability}})
    \item \textbf{Sigurnost} (engl. \textit{\textlatin{safety}})
    \item \textbf{Mogućnost održavanja} (engl. \textit{\textlatin{maintainability}})
\end{enumerate}

\textbf{Dostupnost} će značiti da je sistem spreman \textit{instantno}\footnote{Kada se kaže \textit{instantno}, ne misli se na vremensku mjeru brzine odgovora, već da je sistem spreman dati odgovor u bilo kom vremenskom trenutku, bez vremenskih perioda nemogućnosti obsluživanja.} dati odgovor na neki zahtjev. Generalno gledajući, ova osobina se odnosi na vjerovatnoću da je sistem \textit{pravilno funkcionalan} u bilo kojem momentu - \textit{visoko dostupni sistem je onaj koji će navjerovatnije raditi u datom trenutku}.

\textbf{Pouzdanost} je definisana kroz svojstvo sistema da je u mogućnosti konstantno obsluživati korisnike, bez otkaza. Data osobina može zvučati kao osobina dostupnosti - međutim, pouzdanost je definisana ne u vremenskom trenutku, već u odnosu na vremenski interval. \textit{Visoko pouzdan sistem je onaj koji će najvjerovatnije raditi bez prekida relativno dugo}.

\textbf{Sigurnost} sistem odnosi se na slučaj krahiranja sistema - ako sistem, zbog bilo kojeg razloga, privremeno ne funkcioniše, iz toga neće ishoditi katastrofalne posljedice.

\textbf{Mogućnost održavanja} distribuiranog sistema govori o tome koliko je lako povratiti krahiran sistem \enquote{na noge}. Visoka mogućnost lakog održavanja sistema može ukazati i na visoku pouzdanost, kako sami sistemi mogu biti lako reparirani usljed otkaza.

\subsection{Modeli otkaza}

Do sada smo pričali o osobinama sistema tolerantnog na otkaze, davši i određenu apstrakciju otkaza po kojima se oni dijele po njihovim učestanostima. Međutim, od koristi bi nam bilo i detaljnjije klasifikovati otkaze, na osnovu \textit{njihovog uzroka}. Autori radova na ovu temu definišu nekoliko tipova otkaza koji objašnjavaju njihov uzrok
\begin{itemize}
    \item \textbf{Otkaz usljed krahiranja} dešava se kada čvor prevremeno krahira, ali njegova funkcionalnost nije bila narušena prije toga. Važan aspekat ovog modela jeste da, kada se ovakav tip otkaza desi, čvor više ne odgovara na zahtjeve.
    \item \textbf{Otkaz usljed propusta zahtjeva} modeluje otkaz nastao usljed bezuspješnog odgovora čvora na zahtjev. Pretpostavljajući da je čvor funkcionalan, te da se nije desio otkaz usljed krahiranja, moguće su dvije alternative propusta zahtjeva
    \begin{itemize}
        \item čvor uopšte nije ni dobio zahtjev
        \item čvor jeste dobio zahtjev, ali nije uspio da pošalje odgovor
    \end{itemize}
    \item \textbf{Temporarni otkazi} koji, za razliku od prethodnih otkaza, govore o neuspjehu opsluživanja klijenta u zadatom vremenskom intervalu (koji čak može biti i u realnom vremenu), a što često povezujemo i sa \textit{penalima u performansama sistema}.
\end{itemize}