\section{Konsenzus u distribuiranim sistemima}

Koncept \textbf{zajedničke odluke} u distribuiranim sistemima čest je problem sa kojim se susreće data oblast. Naime, čvorovi u distribuiranom sistemu imaju potrebu \textit{usaglašavanja} (dolaska do \textit{konsenzusa}) zarad \textit{sprovođenja neke akcije} oko \textit{neke vrijednosti}, ma šta ona značila za dati. Međutim, dolazak do konsenzusa u ovakvim sistemima je daleko od lakog zadatka, s obzirom na \textit{otkaze do kojih može doći u sistemu}, a o kojima je govoreno u prethodnom poglavlju.

Fokusiraćemo se na konsenzusu u sistemima sa otkazima bilo kakvog tipa. To će značiti da čvorovi \textit{nisu konzistentni} sa onime što šalju drugim čvorovima - proces \texttt{\textlatin{P}} šalje jednim procesima vrijednost $v$, dok drugim šalje $v'$.